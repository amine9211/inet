\chapter{Node Mobility}
\label{cha:mobility}

\section{Mobility in INET}

\subsection{MobilityBase class}

The abstract \cppclass{MobilityBase} class is the base of the mobility
modules defined in the INET framework. This class implements things like
constraint area (or cubic volume), initial position, and border policy.

When the module is initialized it sets the initial position of the node
by calling the \ffunc{initializePosition()} method. The default implementation
handles the \fpar{initFromDisplayString}, \fpar{initialX}, \fpar{initialY}, \fpar{initialZ}
parameters.

The module is responsible for periodically updating the position.
For this purpose it should send timer messages to itself. These messages
are processed in the \ffunc{handleSelfMessage} method. In derived
classes, \ffunc{handleSelfMessage} should compute the new position
and update the display string and publish the new position by calling
the \ffunc{positionUpdated} method.

When the node reaches the boundary of the constraint area, the mobility
component has to prevent the node to exit. It can call the
\ffunc{handleIfOutside} method, which offers policies like
reflect, torus, random placement, and error.



\subsection{MovingMobilityBase}

The abstract \cppclass{MovingMobilityBase} class can be used to model
mobilities when the node moves on a continous trajectory and
updates its position periodically. Subclasses only need to implement
the \ffunc{move} method that is responsible to update the current
position and speed of the node.

The abstract \ffunc{move} method is called autmotically in every
\fpar{updateInterval} steps. The method is also called when a client
requested the current position or speed or when the \ffunc{move} method
requested an update at a future moment by setting the \fvar{nextChange}
field. This can be used when the state of the motion changes at a
specific time that is not a multiple of \fpar{updateInterval}.
The method can set the \fpar{stationary} field to \ttt{true} to
indicate that the node reached its final position and no more position
update is needed.

% TODO draw a plot of t-position function marking the points when
%      move() is called, stationary set to true, etc.

\subsection{LineSegmentsMobilityBase}

The path of a mobile node often consist of linear movements of constant
speed. The node moves with some speed for some time, then with another
speed for another duration and so on. If a mobility model fits this
description, it might be suitable to derive the implementing C++ class
from \cppclass{LineSegmentsMobilityBase}.

The module first choose a target position and a target time by calling
the \ffunc{setTargetPosition} method. If the target position differs
from the current position, it starts to move toward the target and
updates the position in the configured \fpar{updateInterval} intervals.
When the target position reached, it chooses a new target.

% TODO draw a plot like above, but containing linear segments, mark
%      the points when setTargetPosition called.

% FIXME LineSegmentsMobilityBase should not schedule the self message at updateInterval
%       when lastPosition==targetPosition (the node is waiting at the current position,
%       e.g. every second step in RandomWPMobility)
% TODO Consider an updateInterval computed from an updateDistance and speed, because position change
%      may be irrevelant during a preconfigured updateInterval.

%%% Local Variables:
%%% mode: latex
%%% TeX-master: "usman"
%%% End:


